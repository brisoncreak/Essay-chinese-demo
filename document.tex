\documentclass[UTF8,a4paper,12pt]{ctexart}
\usepackage[left=2.50cm, right=2.50cm, top=2.50cm, bottom=2.50cm]{geometry} %页边距
\CTEXsetup[format={\Large\bfseries}]{section} %设置章标题居左
 
 
%%%%%%%%%%%%%%%%%%%%%%%
% -- text font --
% compile using Xelatex
%%%%%%%%%%%%%%%%%%%%%%%
% -- 中文字体 --
%\setmainfont{Microsoft YaHei}  % 微软雅黑
%\setmainfont{YouYuan}  % 幼圆    
%\setmainfont{NSimSun}  % 新宋体
%\setmainfont{KaiTi}    % 楷体
%\setmainfont{SimSun}   % 宋体
%\setmainfont{SimHei}   % 黑体
% -- 英文字体 --
%\usepackage{times}
%\usepackage{mathpazo}
%\usepackage{fourier}
%\usepackage{charter}
\usepackage{helvet}
 
 
\usepackage{amsmath, amsfonts, amssymb} % math equations, symbols
\usepackage[english]{babel}
\usepackage{color}      % color content
\usepackage{graphicx}   % import figures
\usepackage{url}        % hyperlinks
\usepackage{bm}         % bold type for equations
\usepackage{multirow}
\usepackage{booktabs}
\usepackage{epstopdf}
\usepackage{epsfig}
\usepackage{algorithm}
\usepackage{algorithmic}
\renewcommand{\algorithmicrequire}{ \textbf{Input:}}     % use Input in the format of Algorithm  
\renewcommand{\algorithmicensure}{ \textbf{Initialize:}} % use Initialize in the format of Algorithm  
\renewcommand{\algorithmicreturn}{ \textbf{Output:}}     % use Output in the format of Algorithm  
 

 
\usepackage{fancyhdr} %设置页眉、页脚
\pagestyle{plain}
%\pagestyle{fancy}
\lhead{}
\chead{}
\lfoot{}
\cfoot{}
\rfoot{}
 
 
%%%%%%%%%%%%%%%%%%%%%%%
%  设置水印
%%%%%%%%%%%%%%%%%%%%%%%
%\usepackage{draftwatermark}         % 所有页加水印
%\usepackage[firstpage]{draftwatermark} % 只有第一页加水印
% \SetWatermarkText{Water-Mark}           % 设置水印内容
% \SetWatermarkText{\includegraphics{fig/XXX.eps}}         % 设置水印logo
% \SetWatermarkLightness{0.9}             % 设置水印透明度 0-1
% \SetWatermarkScale{1}                   % 设置水印大小 0-1    
\usepackage{subfig}
 
\usepackage{fancyhdr} %设置页眉、页脚
\pagestyle{plain}
%\pagestyle{fancy}


%参考文献
\usepackage[backend=biber,style=gb7714-2015ay]{biblatex}
\addbibresource[location=local]{bib/sample.bib}


\usepackage[colorlinks,
linkcolor=black,%%修改此处为你想要的颜色
anchorcolor=blue,%%修改此处为你想要的颜色red/blue/green/black/white/cyan/magenta/yellow
citecolor=black,%%修改此处为你想要的颜色,例如修改blue为red
urlcolor=cyan]{hyperref} %bookmarks
%\hypersetup{colorlinks, bookmarks, unicode} %unicode


\usepackage{datetime} %日期
\renewcommand{\today}{\number\year 年 \number\month 月 \number\day 日}

\usepackage[font=small,labelsep=period]{caption} \captionsetup{figurename=图,tablename=表}

\newcommand{\reffig}[1]{(图\ref{#1})}

\newcommand{\degree}{^\circ}




 
 
 
\title{\textbf{小明边的中文小论文}}
\author{ 小明边 \thanks{学号:1903xxxxxx 学院 专业} }
\date{\today}
 
\begin{document}
    \maketitle
     	\maketitle % need full-width title
     	\renewcommand{\abstractname}{}%摘要名为空
     	\begin{abstract}
     		\noindent %摘要无缩进
     		{\bf 摘{} 要:}
     		{\small 这是一篇中文小论文的模版。欢迎大家使用}
     	\end{abstract}

 \section{引言}
 这里是第一章的内容\cite{Liao2018A}。
 \section{实验方法}
 \section{实验结果}
 \subsection{数据}
 \subsection{图表}
 \subsubsection{实验条件}
 \subsubsection{实验过程}
 \subsection{结果分析}	
 \section{结论}
 \section{致谢}
 
 \section*{以下为一些工具}
 
 你好, \LaTeX 。
 
 \begin{enumerate}
 	%$行内公式$$行间公式
 	\item 函数$f(x)$由表达式$$f(x)=3x^2+x-1$$定义,请画出其图像。
 	
 	\item
 	勾股定理:
 	直角三角形的斜边的平方等于两腰的平方和。
 	数学语言表述为:设直角三角形$ABC$,其中 $\angle C=90\degree$,则有:
 	%带编号的公式
 	\begin{equation}
 	AB^2 = BC^2 + AC^2	
 	\end{equation}
 	证明上述定理。
 	
 	\item
 	求矩阵的逆。
 	\[
 	\begin{pmatrix}
 	1 & 2 & 3 \\
 	4 & 5 & 6 \\
 	7 & 8 & 9
 	\end{pmatrix}
 	\]
 	
 	\item
 	求$|A|$
 	\[
 	A = \begin{bmatrix}
 	a_{11} & \dots & a_{1n} \\
 	\         & \ddots & \vdots \\
 	0 & & a_{nn}
 	
 	\end{bmatrix}_{n \times n}
 	\]

 \end{enumerate}
 
 \begin{align}
 & ABCDEFGHIJKLMNOPQRSTUVWXYZ \label{eq:alphabet} \\
 & abcdefghijklmnopqrstuvwxyz \\
 & \alpha \beta \gamma \delta \epsilon \varepsilon \zeta \eta \theta \lambda \mu \nu \xi \pi \rho \sigma \tau \upsilon \phi \varphi \chi \psi \omega  
 \end{align}
 \begin{align}
 \begin{bmatrix}
 1 & 2 \\
 3 & 4 \\
 \end{bmatrix}
 \begin{matrix}
 1 & 2 \\
 3 & 4 \\
 \end{matrix}
 \end{align}
 
 \begin{equation}
 A_{t+1} = \arg\min_A \ \mathcal{L}(A,E_t,\Delta\tau_t,W_t,b_t), \nonumber
 \end{equation}
 
 \begin{equation}
 \begin{aligned} \label{eq:rasl}
 \min_{A,E,\Delta \tau} \quad & \sum_{i=1}^{N}||A_i||_* + \lambda ||E_i||_1  \\
 \mathrm{s.t.} \quad & D_i \circ \tau_i + \sum_{k=1}^{n_i} J_{ik} \Delta \tau_i \epsilon_k \epsilon_k^T = A_i + E_i, \\
 & i = 1,2,\cdots,N. 
 \end{aligned}
 \end{equation}
 
 \begin{table}[htbp]
 	\caption{Title of table} \label{tab:table}
 	\centering
 	\addtolength{\tabcolsep}{-0mm} % 控制列间距
 	\begin{tabular}{ccccc}
 		\toprule[0.75pt]	% package booktabs
 		\multicolumn{4}{c}{table head} \\
 		\midrule[0.5pt]	% package booktabs
 		\multirow{4}{*}{text} & 1 & 2 & 3 & 4 \\  % package multirow
 		& 5 & 6 & 7 & 8 \\
 		\cmidrule[0.5pt]{2-4}	% package booktabs
 		& 9 & 10 & 11 & 12 \\
 		& 13 & 14 & 15 & 16 \\
 		\bottomrule[0.75pt]	% package booktabs
 	\end{tabular}
 \end{table}
 引用: Eq. \eqref{eq:alphabet}, Fig. \ref{figure:fig1},  \\
 
 
 \begin{algorithm}
 	\caption{Title of the Algorithm}
 	\label{algo:ref}
 	\begin{algorithmic}[1]
 		\REQUIRE some words.  % this command shows "Input"
 		\ENSURE ~\\           % this command shows "Initialized"
 		some text goes here ... \\
 		\WHILE {\emph{not converged}}
 		\STATE ... \\  % line number at left side
 		\ENDWHILE
 		\RETURN this is the lat part.  % this command shows "Output"
 	\end{algorithmic}
 \end{algorithm}
 
 ~\\ %空行
 
 插入单张图片\reffig{fig:pair}:
 
  \begin{figure}[htbp]
  	\centering
  	\includegraphics[width=0.55\textwidth]{picture/pair.jpeg} 
  	\caption{结对编程} %caption是图片的标题
  	\label{fig:pair} %此处的label相当于一个图片的专属标志,目的是方便上下文的引用
  \end{figure}
  
 插入浮动图片\reffig{fig:pair-float},第一张\reffig{fig:float-a}和第二张\reffig{fig:float-b}:
 
  \begin{figure}[htbp]
  \subfloat[单图.\label{fig:float-a}]{\includegraphics[width=7cm]{picture/pair.jpeg}}\quad
  \subfloat[浮动.\label{fig:float-b}]{\includegraphics[width=7cm]{picture/pair.jpeg}}\quad
  \caption{结对编程float}
  \label{fig:pair-float}
  \end{figure}
  
  \begin{itemize}
  	\setlength{\itemsep}{0pt}
  	\setlength{\parsep}{0pt}
  	\setlength{\parskip}{0pt}
  	\item[-] 迭代站会不超过15分钟
  	\item[-] 需求描述哦每次不超过1小时
  	\item[-] 展示会1小时以内
  	\item[-] 回顾会不超过两小时
  \end{itemize}
 
\renewcommand\refname{参考文献}
%\bibliomatter
\nocite{*} %打印全部参考文献
\printbibliography
\end{document}
